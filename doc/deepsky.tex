\documentclass{article}
\usepackage{mathtools}

\begin{document}

\section{Task}

We have frames $F(x,y)$:

\begin{eqnarray}
    F(x,y) \sim Poisson\left( \lambda(x,y) \right) \\
    \lambda(x,y) = f_{dark}(x,y) + \nu(x,y) \cdot \left( f_{sky}(x,y) + f_{true}(x,y) \right)
\end{eqnarray}

Where 
\begin{itemize}
    \item $f_{dark}(x,y)$ --- dark signal
    \item $f_{sky}(x,y)$ --- sky signal (light pollution)
    \item $f_{true}(x,y)$ --- true signal
    \item $\nu(x,y)$ --- flat relative calibration function (vignetting, particles on sensor, etc)
\end{itemize}

What does "relative" means? We have telescope lens or mirror, sensor with some QE, exposure, sensor gain. We have Poisson--distributed amount
of photons and dark electrons on sensor, which is multiplied by x,y--independent factor of QE, lens size, exposure, and x,y--dependent factor
of vignetting, particles on sensor, etc. We want to compensate x,y--dependent factors. So we can assume $max(\nu) = 1$, because it shows
relative coefficient to x,y--maximal value. If $max(\nu)$ would be, for example, $0.5$, it would be same as twice less QE with $max(\nu) = 1$.
And we know that it also would be Poisson distribution.

We want to estimate true value of signal: $f_{true}(x,y)$.

\section{Calibrations}

\subsection{Dark signal}

We can find $f_{dark}(x,y)$ by taking enough dark frames $D_k(x,y)$ and find mean of them.
In later research we assume that we know $f_{dark}(x,y)$ with absolute precision.

\begin{equation}
    D_k(x,y) \sim Poisson(f_{dark}(x,y))
\end{equation}

So we find

\begin{equation}
    f_{dark}(x,y) = \frac{\sum_{k=1}^K D_k(x,y)}{K}
\end{equation}

\subsection{Flats}

We need to calculate $\nu(x,y)$. We capture multiple images of sky in different positions:

\begin{eqnarray}
    \Omega_i(x,y) \sim Poisson(\omega_i(x,y)) \\
    \omega_i(x,y) = f_{dark}(x,y) + \nu(x,y) \cdot \left( \tilde f_{sky, i}(x,y) + \tilde f_{value,i}(x,y) \right)
\end{eqnarray}

Where
\begin{itemize}
    \item $\tilde f_{sky, i}(x,y)$ --- sky signal (light pollution) at position $i$
    \item $\tilde f_{value, i}(x,y)$ --- stars signal at position $i$
\end{itemize}

As inputs we have
\begin{itemize}
    \item $f_{dark}(x,y)$ --- dark signal
    \item $\Omega_i(x,y)$ --- flat frames
\end{itemize}

Since 
\begin{equation}
    \Omega_i(x,y) \sim Poisson(\omega_i(x,y)) = Poisson(f_{dark}(x,y) + \nu(x,y) \cdot \left( \tilde f_i(x,y) \right))
\end{equation}

We find mean $<\Omega_i>$:
\begin{eqnarray}
    <\Omega_i(x,y)> = \frac{\sum \Omega_i(x,y)}{N} = f_{dark}(x,y) + \nu(x,y) \cdot \frac{\tilde f_{sky, i}(x,y) + \tilde f_{value,i}(x,y)}{N} \\
    \nu(x,y) = \frac{\frac{\sum \Omega_i(x,y)}{N} - f_{dark}(x,y)}{\frac{\sum \tilde f_{sky, i}(x,y) + \tilde f_{value,i}(x,y)}{N}}
    = \frac{< \Omega_i(x,y) - f_{dark}(x,y) >}{<\tilde f_{sky, i}(x,y) + \tilde f_{value,i}(x,y)>}
\end{eqnarray}

Now we need to estimate both parts of this fractions. Since we have random selections of areas on each $i$, we can assume that
$<\tilde f_{sky, i}(x,y) + \tilde f_{value,i}(x,y)> = const$. Also we can use Sigma-Clipping algorithm to reduce amount of images $\Omega_i$.

Also we expect $\nu$ be so that $max(\nu) = 1$. As result we have:

\begin{eqnarray}
    \nu(x,y) = \frac{SC(\left\{ \Omega_i(x,y) - f_{dark}(x,y) \right\})}{max\left(SC(\left\{ \Omega_i(x,y) - f_{dark}(x,y) \right\})\right)}
\end{eqnarray}

After this we can denoise obtained $\nu(x,y)$ with blur or other methods.

\section{Signal obtaining}

As input we have:
\begin{itemize}
    \item $f_{dark}(x,y)$ --- dark signal
    \item $\nu(x,y)$ --- normalized flats calibration function
    \item $F_j(x,y)$ --- frames
\end{itemize}

Our whole signal $f(x,y)$ contains of multiple stages. In simple case we have just sky light pollution:
\begin{eqnarray}
    f(x,y) = f_{sky}(x,y) + f_{signal}(x,y)
\end{eqnarray}

In more complex cases when we want to obtain narrowband signal we have continuum in addition to sky:
\begin{eqnarray}
    f(x,y) = f_{sky}(x,y) + f_{continuum}(x,y) + f_{narrow}(x,y)
\end{eqnarray}

But on this stage our task is to find $f(x,y)$.

We have frames $F_i$:
\begin{eqnarray}
    F_i(x,y) = Poisson\left( f_{dark}(x,y) + \nu(x,y) \cdot f(x,y) \right)
\end{eqnarray}

\subsection{First approach}

We act the same way as for flats calculation:
\begin{eqnarray}
    <F_i(x,y)> = f_{dark}(x,y) + \nu(x,y) \cdot f(x,y) \\
    f(x,y) = \frac{<F_i(x,y)> - f_{dark}(x,y)}{\nu(x,y)} = \frac{<F_i(x,y) - f_{dark}(x,y)>}{\nu(x,y)}
\end{eqnarray}

Here we use Sigma-Clipping:

\begin{eqnarray}
    f(x,y) = \frac{SC(\left\{ F_i(x,y) - f_{dark}(x,y)\right\})}{\nu(x,y)}
\end{eqnarray}

This implements the most basic approach: substract darks, then apply flats, then do Sigma-Clipping.

This method is simple, but can generate negative values, especially for low SNR case.

\subsection{Bayesian analysis}

We have set of frames $F_i(x,y)$. We want to find estimation for $f(x,y)$:

\begin{eqnarray}
    E\left[ f(x,y) \right] = \int df(x,y) f(x,y) p(f(x,y) | \left\{ F_i(x,y) \right\})
\end{eqnarray}

Let's find $p(f(x,y) | \left\{ F_i(x,y) \right\})$. We process independently for each $x,y$, so they are just parameters,
not integration varables. So we won't write them for readability:

\begin{eqnarray}
    E\left[ f \right] = \int df f p(f | \left\{ F_i \right\})
\end{eqnarray}

According to Bayes' theorem

\begin{eqnarray}
    p(f | \left\{ F_i \right\}) = \frac{p(\left\{ F_i \right\} | f) \cdot p(f)}{p(\left\{ F_i \right\})} =
    \frac{p(\left\{ F_i \right\} | f) \cdot p(f)}{\int p(\left\{ F_i \right\} | f') \cdot p(f')} df' \label{eq:bayes}
\end{eqnarray}

where $p(f)$ --- apriori probability of $f$. We should estimate it before starting Bayesian analysis.

It's very important that
\begin{equation}
    p(\left\{ F_i \right\}) \ne \prod_i p(F_i)
\end{equation}

It happens because $F_i$ are not marginally independent: if $P(AB|C_i) = P(A|C_i) \cdot P(B|C_i)$ for each $i$, it doesn't mean that $P(AB) = P(A)\cdot P(B)$

But $F_i|f$ are conditionally independent, so we can write
\begin{equation}
    p(\left\{ F_i \right\} | f) = \prod_i p(F_i | f)
\end{equation}

So we can continue (\ref{eq:bayes}) and write it as
\begin{eqnarray}
    p(f | \left\{ F_i \right\}) = \frac{\prod_i p(F_i|f) \cdot p(f)}{\int \prod_i p(F_i|f') \cdot p(f') df'} \label{eq:bayes2}
\end{eqnarray}

Let's remember what $p(F_i | f)$ is:
\begin{equation}
\begin{array}{r@{}l}
    p(F_i | f) = \frac{\lambda^{F_i} e^{-\lambda}} {F_i!} \\
    \lambda = f_{dark}(x,y) + \nu(x,y) f(x,y)
\end{array}
\end{equation}

Let's substutute it into (\ref{eq:bayes2}):
\begin{equation}
\begin{array}{r@{}l}
    p(f | \left\{ F_i \right\}) = \frac{\prod_i \frac{\lambda(f)^{F_i} \cdot e^{-\lambda(f)}}{F_i!} \cdot p(f)}
    {\int \prod_i \frac{\lambda(f)^{F_i} \cdot e^{-\lambda(f)}}{F_i!} p(f') df'}
\end{array}
\end{equation}

After simplyfying it we have:
\begin{equation}
    p(f | \left\{ F_i \right\}) = \frac{\lambda(f)^{\sum F_i} \cdot e^{-N\lambda(f)} \cdot p(f)}
    {\int \lambda(f')^{\sum F_i} \cdot e^{-N\lambda(f')} \cdot p(f') df'}
\end{equation}

where $N$ --- amount of $F_i$. Factorials are gone. And for case where $p(f) \ne 0$, $\lambda(f) \ne 0$ we can simplify:

\begin{equation}
    p(f | \left\{ F_i \right\}) = \left( \int \left( \frac{\lambda(f')}{\lambda(f)} \right)^{\sum F_i} \cdot e^{-N(\lambda(f')-\lambda(f))} \cdot \left( \frac{p(f')}{p(f)} \right) df'\right)^{-1}
\end{equation}

But there can be some practical case: what if different frames have different $f_{dark}$ or $f_{sky}$ or $\nu$? In that case we have series
of $\lambda_i(f)$ and formula slightly changes:
\begin{equation}
\begin{array}{r@{}l}
    E[f]\left( \left\{ F_i \right\} \right) = \int f \cdot p(f | {F_i}) df \\
    p(f | \left\{ F_i \right\}) = \left( \int \prod_i \left( \frac{\lambda_i(f')}{\lambda_i(f)} \right)^{F_i} \cdot 
    e^{-\sum_i \left(\lambda_i(f')-\lambda_i(f)\right)} \cdot \left( \frac{p(f')}{p(f)} \right) df'\right)^{-1}
\end{array}
\end{equation}

In this formul wie have 2 components $\prod_i \left( \frac{\lambda_i(f')}{\lambda_i(f)} \right)^{F_i}$ and
$e^{-\sum_i \left(\lambda_i(f')-\lambda_i(f)\right)}$. For big $f'$ first would be big, but it will be compensated by exponent.
But it can cause overflow. So we can write:

\begin{eqnarray}
    p(f | \left\{ F_i \right\}) = \left( \int e^{\sum_i \left( F_i ln \lambda_i(f') - \lambda_i(f')\right) 
            - \left( F_i ln \lambda_i(f) - \lambda_i(f)\right)} \cdot \frac{p(f')}{p(f)} df' \right)^{-1} \label{eq:bayes3} \\[9pt]
    \Lambda_i(f) = F_i ln \lambda_i(f) - \lambda_i(f) \\[9pt]
    p(f | \left\{ F_i \right\}) = \left( \int e^{\sum_i \Lambda_i(f') - \Lambda_i(f) }\cdot \frac{p(f')}{p(f)} df' \right)^{-1}
\end{eqnarray}



Challenges with such approach:
\begin{itemize}
\item we need some a-priori knowledge of probability $p(f)$
\item integral calculation over $R_{+}$ is pretty long procedure. So we can calculate it only in some area around maxima of $p(f')$
\end{itemize}

Good point is that for some priors, integral can be taken analytically.

\subsection{Pipeline}
So the pipeline is following:

\begin{itemize}
    \item Perform a "naive" estimation to obtain initial $f$
    \item Using a sliding window, build local histograms of $f$ to estimate the local prior $p(f)$. Normalize and optionally smooth these histograms.
    \item Use the Bayesian estimator for $E\left[ f \right]$ to refine $f$, based on the current prior
    \item Iterate the prior estimation and refinement steps until convergence or acceptable error level
\end{itemize}

\section{True signal obtain}

During previous step we studied how to obtain signal $f$. But it contains sky light pollution:
\begin{eqnarray}
    f = f_{sky} + f_{signal} \\
    \lambda(f_{signal}) = f_{dark} + \nu \left( f_{sky} + f_{signal} \right)
\end{eqnarray}

So, we need to estimate $f_{sky}$.

\subsection{Sky estimation}
Firstly, let's using formula $\lambda(f) = f_{dark} + \nu f$ find whole captured signal $f$.

We can estimate $f_{sky}$ from $f$ using regions on the side of image, where target signal is expected to be absent.
But we should be careful, not to overestimate.

\subsection{Bayesian analysis}

So, now we use $\lambda(f_{signal}) = f_{dark} + \nu \left( f_{sky} + f_{signal} \right)$ to find $f_{signal}$.

\subsection{Pipeline}
So the pipeline is following:

\begin{itemize}
    \item obtain whole signal $f(x,y)$
    \item find $f_{sky}$ from $f(x,y)$
    \item find expected $f_{signal}$
\end{itemize}

\section{Narrowband signal extraction}

In this case we have 2 signal images, captured with narrow (1) and wide (2) filters. Whole signal has 2 components: $f_{c,*}$ --- continuum,
and $f_{n,*}$ --- narrow signal:

\begin{eqnarray}
    f_{signal, 1} = f_{c, 1} + f_{n} \\
    f_{signal, 2} = f_{c, 2} + f_{n}
\end{eqnarray}

And we assume that continuum for both cases is proportional. We can assume that if wide filter not too wide.
\begin{equation}
    f_{c,2} = k \cdot f_{c,1} = k \cdot f_c
\end{equation}, where $k > 1$. Value $k$ we obtain from calibration by stars --- they emit only continuum.

So
\begin{eqnarray}
    f_{signal, 1} = f_c + f_n \\
    f_{signal, 2} = k \cdot f_c + f_n
\end{eqnarray}

So we can find $f_n$ as
\begin{eqnarray}
    (1-1/k) \cdot f_n = f_{signal, 1} - \frac{f_{signal, 2}}{k} \\
    f_n = \frac{k \cdot f_{signal, 1} - f_{signal, 2}}{k-1} \\
    f_c = f_{signal,1} - f_n = \frac{f_{signal, 2} - f_{signal, 1}}{k-1}
\end{eqnarray}

But as in previous parts, it's also a first approach.

\subsection{Bayesian analysis}

\begin{eqnarray}
    F_{i,1} = Poisson\left( \lambda_{i,1}\left({f_n \atop f_c}\right) \right) \\
    F_{j,2} = Poisson\left( \lambda_{j,2}\left({f_n \atop f_c}\right) \right)
\end{eqnarray}

\begin{eqnarray}
    \lambda_{i,1}\left({f_n \atop f_c}\right) = f_{dark, 1, i} + \nu_{1,i} \cdot \left( f_{sky, 1, i} + f_c + f_n \right) \\
    \lambda_{j,2}\left({f_n \atop f_c}\right) = f_{dark, 2, j} + \nu_{2,j} \cdot \left( f_{sky, 2, j} + k \cdot f_c + f_n \right)
\end{eqnarray}

We can concatinate $\left\{F_{i,1}, F_{j,2}\right\}$ frames to single frames list and
$\left\{ \lambda_{i,1}, \lambda_{j,2} \right\}$ to single $\lambda$ list, and like in (\ref{eq:bayes3}) we get:

\begin{eqnarray}
p\left({f_n \atop f_c} | \left\{ F_{i,1}, F_{j,2} \right\}\right) =
\left( \int e^{
    -\sum_i \left[\Lambda_{i,1}\left({f'_n \atop f'_c}\right) - \Lambda_{i,1}\left({f_n \atop f_c}\right) \right]
    -\sum_j \left[\Lambda_{j,2}\left({f'_n \atop f'_c}\right) - \Lambda_{j,2}\left({f_n \atop f_c}\right) \right]}
\cdot \frac{p\left({f'_n \atop f'_c}\right)}{p\left({f_n \atop f_c}\right)} df'_n df'_c \right)^{-1}
\end{eqnarray}

\end{document}
